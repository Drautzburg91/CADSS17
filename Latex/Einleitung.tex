\section{Einleitung}

Im heutigen Zeitalter gewinnen verteilte Systeme immer mehr an Bedeutung. Um zeitgerechten Performanceanforderungen gerecht zu werden,  setzen viele Entwickler und Firmen auf Cloud Architektur Lösungen, um ihre Produkte und Softwarelösung für Kunden zugänglich zu machen.
Verschiedene Anbieter haben sich in den vergangenen Jahren etabliert und bieten inzwischen eine Reihe von Services an. Obwohl sich der Rahmen der Cloud Angebote stark unterscheidet, lassen sich die Vorteile des Cloud Computing im Allgemeinen herausstellen. Neben Rechen- und Speicherkapazität, lassen sich auch Dienste anmieten, wodurch eine längerfristige Kapitalbindung für benötigte Hardware vermieden werden kann. Des Weiteren werden die Hardwarekomponenten durch den Cloud Anbieter in der Regel auf dem neusten Stand der Technik gehalten, was für viele Unternehmensstrukturen im Vergleich zu eigener Hardware rentabel ist.  Je nach Vertragsabkommen kann man auch auf die IT Expertise der Cloud Anbieter zurückgreifen und verringert somit die Abhängigkeit von eigenen IT-Mitarbeitern für Aufbereitung und Instandhaltung seines Systems. Zum Beispiel kann die Verantwortung für Performanceattribute wie unterbrechungsfreie Strom Versorgung (USV), Security und SLA auf den Cloud Anbieter übertragen werden.  Der wertvollste Vorteil einer Cloud Architektur liegt in der Skalierbarkeit der Dienste und der zugrunde liegenden Hardware. Über konfigurierbare Einstellungen, lassen sich die Systemkomponenten ja nach Nutzungsgrad skalieren. Dies ermöglicht schnelle Reaktionen auf Wachstum oder Nutzungsspitzen. Das Angebot lässt sich in die Bereiche Infrastructure as a Service (IaaS), Platform as a Service (PaaS) und Software as a Service  (SaaS) unterteilen, welche in einer Vielzahl von Variationen zur Verfügung gestellt werden.
Um zukünftigen, beruflichen Aufgaben eines Cloud Entwicklers gerecht zu werden, wird in der Fachrichtung Software Engineering des Masterstudiengangs Informatik an der HTWG das Fach Cloud Application Development angeboten. Der Kurs soll helfen die komplexen Strukturen eines Cloud Systems kennenzulernen und zu beherrschen. Im Rahmen einer Projektarbeit sollen wir Studierenden den Umgang mit den Architekturelementen erlernen. Nach der Definition eines geeigneten Anwendungsszenarios sollen die Studierenden eine funktionierende Cloud Architektur einrichten und testen.
Diese Dokumentation beschreibt die Umsetzung der Projektarbeit und ist Teil der Bewertung für das Studienmodul Cloud Application Development. Neben einer ersten Erfahrung sind die Studierenden in der Lage Angebote und Dienste im Bereich Cloud Computing differenziert zu betrachten und anhand entscheidender Kriterien für unterschiedliche Anwendungsfälle zu bewerten. 





