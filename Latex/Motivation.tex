\section{Motivation}
\subsection{Zielsetzung}
ToDo: Verweis auf 12 Faktor APP Standard !!! 
Die Tabelle soll am Anfang stehen, am besten in der Zielsetzung.
Die folgende Tabelle beschreibt die Kernanforderungen der 12 Faktor APP, 
\begin{table}[!ht]
  \centering
    \begin{minipage}{15cm}
      \centering
      \begin{tabular}{*{3}{|l|p{5.0cm}|p{5.0cm}}}\hline
      \multicolumn{3}{|c|}{\cellcolor[RGB]{200,200,200}12 Faktor APP Anforderungen} \\\hline
     \textbf{ID}&\textbf{Anforderung}&\textbf{Beschreibung}\\\hline
     1.&Codebase&Eine im Versionsmanagementsystem verwaltete Codebase, viele Deployments.\\
      \hline
     2.&Abhängigkeiten&Abhängigkeiten explizit deklarieren und isolieren.\\
     \hline
     3.&Konfiguration&Die Konfiguration in Umgebungsvariablen ablegen.\\
     \hline
     4.&Unterstützende Dienste&Unterstützende Dienste als angehängte Ressourcen behandeln.\\
     \hline 
     5.&Build, release, run&Build- und Run-Phase strikt trennen.\\
     \hline
     6.&Prozesse&Die App als einen oder mehrere Prozesse ausführen.\\
     \hline
      7.&Bindung an Ports&Dienste durch das Binden von Ports exportieren.\\
     \hline
      8.&Nebenläufigkeit&Mit dem Prozess-Modell skalieren.\\
     \hline
      9.&Einweggebrauch&Robuster mit schnellem Start und problemlosen Stopp.\\
     \hline
     10.&Dev-Prod-Vergleichbarkeit&Entwicklung, Staging und Produktion so ähnlich wie möglich halten.\\
     \hline     
     11.&Logs&Logs als Strom von Ereignissen behandeln.\\
     \hline
     12.&Admin-Prozesse&Admin/Management-Aufgaben als einmalige Vorgänge behandeln.\\
     \hline
      \end{tabular}
   \caption{12 Faktor App Anforderungen}\label{tab:Anforderungen}
    \end{minipage}
\end{table}

\clearpage

\subsection{Die 12 Faktor-APP Anforderungen}