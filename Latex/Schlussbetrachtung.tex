\section{Schlussbetrachtung}
In diesem letzten Kapitel wird zu Beginn ein Fazit gezogen, in welchem grob auf die Eindrücke, Lerneffekte und Erkenntnisse eingegangen wird. Im abschließenden Abschnitt wird ein Ausblick auf mögliche Erweiterungen und Verbesserungen des Systems gegeben. 
\subsection{Fazit}

Durch die große Anzahl an Teammitgliedern war von Beginn an ein hoher Aufwand an Projektmanagement erforderlich. Aus diesem Grund wurde das Vorgehensmodell Scrum verwendet und in diesem Rahmen ein Jira-Board angelegt. Der Einsatz eines Vorgehensmodells gestaltete sich als Schwierig, da keine Rollen definiert wurden, so sollte ein Scrum-Team mind. eine Teamleiterrolle beinhalten. Durch diese Tatsache wurde aufgezeigt, dass eine klare Struktur in einem in einem großen Team unerlässlich ist, sei es durch die Definition von Rollen oder Verantwortlichkeiten. 
Im Laufe des Projektes besserte sich die angesprochene Situation jedoch. Da sich durch die verschiedenen Aufgabenstellungen, welche sich auf Grund des Projektes ergaben, zwangsläufig Verantwortlichkeiten heraus kristallisierten. 
Daraus resultierten sich dann eine produktive Teamleistung. 

Der zweite nicht außer acht zu lassende Aufwand, ist die Integration der einzelnen Komponenten. 
Da der Projektumfang durch die Anzahl der Teammitglieder gestiegen war, erwies sich die Integration der Einzelkomponenten aufwendiger als geplant. Dabei spielte das Deployment in eine Cloudumgebung eine wesentliche Rolle. Da sich dieses vom herkömmlichen, uns bekannten, on premise-Deployment unterschied. Diese Tatsache wurde durch den Einsatz eines automatisierten Builds durch einen Jenkins-Server noch weiter verkompliziert. 
Diese daraus gewonnen Erkenntnisse waren ein wichtiger Aspekt für das gesamte Team, da die Erfahrungen in einer solchen Umgebung für alle Neuland war. 

Außerdem waren die Anforderungen der "12-Faktor-App" für den Rahmen dieser Arbeit von hoher Wichtigkeit. Aufgrund dieser Anforderungen an die Lösung, mussten die einzelnen Komponenten nachträglich angepasst werden. Als ein großer positiver Aspekt kann bei den 12 Faktoren die gute Strukturierung und die reine Ausrichtung auf Cloud Applikationen genannt werden. Dies ermöglichte einen relativ problemlosen Übergang von lokalem Entwicklungssystem auf die einzelnen Cloud Dienste. 
\clearpage
\subsection{Ausblick}

Für die Zukunft des MOM based Information Life Flow Systems gibt es einige Erweiterungsmöglichkeiten. Dazu gehören in erster Linie die Komponenten, Complex Event Process und die Datenbank. Da alle Wetterdaten gespeichert werden, könnten in der Zukunft bspw. auf historische Daten zugriffen werden um Statistiken zu erstellen und ein Regenrisiko für einen Tag zu bestimmen. Außerdem können auf Grundlage dieser Daten auch weitere Events-Szenarien entwickelt werden. Zudem könnten die Weather-Tenants erweitert werden um bspw. eine Differenzierung der bereitgestellten Daten für verschiedene Benutzergruppen zu gewährleisten. Somit könnte ein Modell auf Basis von "Premium" und "Standard" Mitgliedschaften implementiert werden.
