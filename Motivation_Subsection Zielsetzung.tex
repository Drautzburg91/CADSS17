\section{Zielsetzung}
In modernen Systemarchitekturen wird Software vermehrt als Dienst zur Verfügung gestellt, was in Bezug auf Cloud Lösungen oft als Software-As-A-Service (SaaS) bezeichnet wird.  Um stetig komplexen Anforderungen mit unterschiedlichen personellen Entwicklerorganisationen  gerecht zu werden, existiert die sogenannte \textbf{Twelve-Factor App}. Hinter dem Begriff verbirgt sich ein Konzept um SaaS Applikationen strukturiert nach 12 Faktoren bzw. Kategorien aufzubauen. Die Kriterien einer 12-Factor App wurden aus den Erfahrungswerten vieler Projekte erarbeitet. Neben grundlegenden Terminologien, Entwicklungs- und Konfigurationschritten, werden auch Lösungsstrategien für potentielle Problemstellungen beschrieben. Zum Beispiel sind verschiedene Eigenschaften wie Continuous Deployment, maximale Portierbarkeit und Skalierung berücksichtigt, um das dynamische Verhalten einer SaaS Lösung effizient aufzufangen.\\ \\


\begin{itshape}
„Unsere Motivation ist, das Bewusstsein zu schärfen für systembedingte Probleme in der aktuellen Applikationsentwicklung, ein gemeinsames Vokabular zur Diskussion dieser Probleme zu liefern und ein Lösungsportfolio zu diesen Problemen mit einer zugehörigen Terminologie anzubieten. Das Format ist angelehnt an Martin Fowlers Bücher Patterns of Enterprise Application Architecture und Refactoring.“ 
\end{itshape} 
[Quelle: https://12factor.net/de/] \\ \\

Als Teil der Prüfungskriterien im Fach Cloud Application Development, betrachten wir die Umsetzung unseres Teamprojektes nach der Methodik der 12-Factor App. Hierzu wird in den jeweiligen Kapiteln der einzelnen Komponenten die nachfolgende Tabelle referenziert, welche die Faktoren geordnet aufzeigt: